\chapter{Revisioni delle stime precedenti}\label{cha:rdsp}
\minitoc\mtcskip

Dopo le considerazioni effettuate nel Capitolo \vref{cha:pds}, possiamo effettuare
una prima revisione delle stime che sono state precedentemente effettuate (v.
Sezione \vref{sec:mfsvds}).

\section{Aggiunta di Tool di sviluppo}
Si è introdotto \textsc{Inkscape} per la compilazione dei files di estensione 
\textsc{svg} in \textsc{pdf}, e per modifiche grafiche ai diagrammi ottenuti 
con \textsc{ArgoUML}

\section{Revisione della valutazione dello sforzo}

\subsection{Valutazione dell'analisi dello sforzo con gli Use Case}
Si potrebbe pensare di effettuare una nuova valutazione della stima dello sforzo
tramite l'impiego degli Use Case; di fatti, da questi use case, dipendono i 
seguenti elementi di progettazione:
\begin{itemize}
\diam Dimensione dell'applicazione in LOC
\diam Numero di test case prodotti
\end{itemize}
Tuttavia, in quanto questi Use Case possono essere creati a differenti livelli
di astrazione, non ci sentiamo di utilizzare questa tipologia di stima in sostituzione
all'analisi precedente\footnote{R.S. Pressman: op. cit.}. Inoltre,
non è nemmeno possibile utilizzare la formula di Smith\footnote{R.S. Pressman: 
op. cit.}, in quanto non possediamo dati medi o/e storici in LOC per 
use case di questo tipo di sistema, necessari al computo richiesto dalla formula.

\subsection{Revisione dei computi effettuati precedentemente}
In considerazione di quanto rilevato precedentemente, ed
in seguito alla modifica dei requisiti apportata come conseguenza dell'incontro
nel cliente, descritto nella Sezione \vref{sec:dsicc}, possiamo effettuare una
revisione della stima dello sforzo rispetto alla valutazione fornita nel 
\textbf{Documento Primo}.

\begin{table}[!t]
\centering
\begin{tabular}{cccc}
\toprule
\textbf{Unadjusted Function Point (UFP)} & \textbf{Tipologia} & \textbf{Quantità}\\
\midrule
\textit{Input (EI)} & Semplice & 2\\
\textit{Input (EI)} & Media & 3\\
\textit{Output (EO)} & Semplice & 1\\
\textit{Output (EO)} & Media & 2\\
\textit{Interrogazioni (EQ)} & Semplice & 8\\
\textit{File logici interni (ILF)} & - & 0\\
\textit{File logici esterni (EIF)} & Semplice & 1\\
\bottomrule
Totale & & 55\\
\end{tabular}
\caption{\textit{Definizione degli} Unadjusted Function Point \textit{: seconda stima}.}
\label{tab:tdd5}
\end{table}

Rieffettuando la computazione delle stime precedentemente effettuate, abbiamo
che l'unica variazione degna di nota si è riscontrata negli UFP, dei quali 
rieffettuiamo la stima all'interno della Tabella \vref{tab:tdd5}, dovuta 
al fatto che si è preferito unire interfacce grafiche con compiti simili.
\[\begin{split}
FP &= conteggio-UFP \cdot (0.65+0.01\cdot\sum_{i=1}^{14} F_i)\\
   &= 55\cdot (0.65+0.01\cdot 26)=50.05
\end{split}\]
Considerando ancora il fattore di conversione FP/LOC con il linguaggio Java,
otteniamo:
\[LOC = 63\cdot 66=3153\]
Da cui dalla stima di \textit{Bailey-Basili}\footnote{R.S. Pressman: op. cit.}
otteniamo:
\[\begin{split}
E_{bb} &= 5,5+0,7\cdot {KLOC}^{1,16} mesi/persona\\
       &= 5.5+0.7\cdot {4,3}^{1.16}\simeq 8.15 mesi/persona
\end{split}\]
Si osserva quindi una riduzione dei costi, basata unicamente dalla riduzione
della stima effettuata dalla valutazione dei dati transitanti dell'interfaccia:
tuttavia non reputiamo questa stima soddisfacente per la valutazione della
complessità intrinseca del progetto.
