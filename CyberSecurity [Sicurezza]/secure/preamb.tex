%% preambolo standard dei miei documenti %%
\documentclass[a4paper,12pt,italian]{book}

\usepackage{makeidx}
\usepackage{amsfonts}
\usepackage{amsmath,amsthm}
\usepackage{mathpazo}
\usepackage[tight,italian]{minitoc}
\usepackage{braket}
\usepackage[T1]{fontenc}
\usepackage[utf8]{inputenc}
\usepackage[a4paper]{geometry}
\geometry{verbose,tmargin=2cm,bmargin=3cm,lmargin=2cm,rmargin=2cm,headheight=2cm,headsep=1cm,footskip=1cm}
\usepackage{amsthm}
\usepackage[italian]{babel}


\usepackage{listings} %­Per inserire codice
\usepackage[usenames]{color} %­Per permettere la colorazione dei caratteri


\usepackage[titles]{tocloft}

%% Aesthetic spacing redefines that look nicer to me than the defaults.

\setlength{\cftbeforechapskip}{2ex}
\setlength{\cftbeforesecskip}{0.5ex}

%% Use Helvetica-Narrow Bold for Chapter entries

\renewcommand{\cftchapfont}{%
  \fontsize{11}{13}\usefont{OT1}{phv}{bc}{n}\selectfont
}



% Impostazioni per lstlistning
\lstdefinelanguage{pseudoc}
{
morekeywords = {else, return, if, for, printf, do, while, int, in, foreach, long, raise, void, NULL, switch, case, then, match, with, typedef, struct},
basicstyle=\small\sffamily,
morecomment=[l][\color{black}] {//}, % colore commento // ...
morecomment=[s][\color{black}]{/*}{*/}, % colore commento /* ... */
morestring=[b][\color{red}]", % colore stringhe
showstringspaces=false,
frame = trBL,
columns=fixed,
mathescape=true,
keywordstyle=\color{black}\bfseries % parole chiave in blue e in grassetto
}
\lstdefinelanguage{OCaml}
{
morekeywords = {let, rec, in, match, with, ref, if, then, else},
basicstyle=\small\sffamily,
morecomment=[l][\color{black}] {//}, % colore commento // ...
morecomment=[s][\color{black}]{/*}{*/}, % colore commento /* ... */
morestring=[b][\color{red}]", % colore stringhe
showstringspaces=false,
frame = trBL,
columns=fixed,
mathescape=true,
keywordstyle=\color{black}\bfseries % parole chiave in blue e in grassetto
}



\lstnewenvironment{pseudoc}{
\lstset{language=pseudoc}}{}

\lstnewenvironment{OCaml}{
\lstset{language=OCaml}}{}

\theoremstyle{plain}\newtheorem{teorema}{Teorema}[chapter]
\theoremstyle{plain}\newtheorem{corollario}{Corollario}[chapter]
\theoremstyle{plain}\newtheorem{define}{Definizione}
\makeatother

\usepackage{booktabs}
\usepackage{tabularx}
\usepackage{rotating}
\usepackage{graphicx}
\usepackage[italian]{varioref}
\usepackage{subfig}
\setcounter{secnumdepth}{9}
\setcounter{tocdepth}{9}
\usepackage[unicode=true, pdfusetitle,
 bookmarks=true,bookmarksnumbered=true,bookmarksopen=true,bookmarksopenlevel=1,
 pdfborder={0 0 1},backref=false,colorlinks=false]
 {hyperref}

% mie definizoni 
\newcommand{\diam}{\item[$\diamond$]} % punto elenco diamante
\newcommand{\cerc}{\item[$\circ$]} % punto elenco cerchio
\newcommand{\suspence}{[\dots]} % ellissi
\newcommand{\opquote}{«}
\newcommand{\clquote}{»}
\newcommand{\reffig}{Figura}
\newcommand{\singlediam}[1]{\begin{itemize}\diam #1\end{itemize}} % un solo punto elenco con diamante
\newcommand{\ttilde}{\textasciitilde{}  } % scrittura semplificata della tilde
\renewcommand{\*}{\cdot}
\newcommand{\myquote}[2]{«\emph{#1}»(#2)}
\newcommand{\myquotu}[1]{«\emph{#1}»}
\newcommand{\textbb}[1]{$\mathbb{#1}$}

 % definizione dell'elenco con le lettere (wikipedia.en)
%\arabic 	1, 2, 3 ...
%\alph 	a, b, c ...
%\Alph 	A, B, C ...
%\roman 	i, ii, iii ...
%\Roman 	I, II, III ...
%\fnsymbol 	Aimed at footnotes (see below), but prints a sequence of symbols.
 % come consigliato dalla Nasa (http://www.giss.nasa.gov/tools/latex/ltx-222.html):
 % \renewcommand{\labelenumi}{\emph{(\alph{enumi})}}
\newenvironment{itmletters}[1]{\begin{enumerate}\renewcommand{\labelenumi}{\emph{(#1)}}}
                              {\end{enumerate}}

 % questo comando crea l'elenco con i letterali definiti sopra
\newcommand{\atm}{\alph{enumi}}
 % questo comando serve per indicare in quello sopra l'enumerazione della prima serie di numeri
 % come di 
\newcommand{\itm}{\roman{enumi}}
\newcommand{\cent}[1]{\begin{center}#1\end{center}}
\newcommand{\textcal}[1]{$\mathcal{#1}$}
              \newcommand{\todo}{\begin{center}\end{center}}          
                        
\newcommand{\phsubs}[1]{\subsection*{#1}\addcontentsline{toc}{subsection}{#1}}
\newcommand{\phsubssubs}[1]{\subsubsection*{#1}\addcontentsline{toc}{subsubsection}{#1}}
\newcommand{\todoing}{\begin{center}\bf{todo}\end{center}}

                        
\newcommand{\startfrom}[1]{\setcounter{enumi}{#1}}
\newcommand{\ttitem}[1]{\item \texttt{#1}}
\newcommand{\tleq}{$\leq$} 
\newcommand{\tgeq}{$\geq$}
\newtheorem{theorem}{Teorema}
\newenvironment{teordim}[1]{\begin{theorem}#1\end{theorem}\begin{proof}}{\end{proof}}
%\newenvironment{mkitem}[1]{\begin{#1}}{\end{#1}}

\newtheorem{example}{Esempio}
\newtheorem{exercise}{Esercizio}
\newtheorem{prop}{Proposizione}
\newtheorem{defin}{Definizione}
\newtheorem{oss}{Osservazione}


\newcommand{\real}{\mathbb{R}}


%% Aesthetic spacing redefines that look nicer to me than the defaults.
\setlength{\parskip}{0pt}
\setlength{\parsep}{0pt}

\setlength{\partopsep}{0pt}



\linespread{1}



