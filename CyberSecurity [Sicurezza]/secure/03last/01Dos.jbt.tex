\chapter{Gestione delle Reti}
\section{Denial of Service (DoS)}
La minaccia maggiore per i sistemi informatici è quella della mancata fruizione
di un servizio mediante ``cyberextorsion'', mirando ad una perdita economica
da parte di chi detiene la fornitura del servizio stesso. Mentre nel mondo
fisico questo metodo di attacco è molto ridotto perché per effettuare SPAM, è
necessario sobbarcarsi dei costi troppo ingenti, questo non avviene nel mondo
informatico, dove i costi sono molto ridotti se non nulli. Per risolvere un
potenziale attacco si potrebbero effettuare le seguenti considerazioni:
\begin{itemize}
\diam Si potrebbe cercare di evitare l'attacco mediante l'utilizzo di \textit{blacklist},
	non consentendo più l'accesso ad un dato utente: tuttavia in questo
	caso potrei escludere un accesso legittimo di un utente con (es.) lo
	stesso numero di IP.
\diam Potrei cassare una richiesta onerosa, anche se alcune potrebbero essere
	in buona fede: si potrebbe quindi richiedere al potenziale attaccante
	un ``costo'' sufficientemente elevato da demotivarlo nel perseguire
	l'attacco.
	
	Tuttavia questa situazione potrebbe essere aggirata
	richiedendo chiedendo a tanti clienti
	di effettuare moltissime richieste che stiano al di sotto della 
	soglia d'allarme. Questo tipo d'attacco è chiamato un DDoS, ovvero un
	distributed DoS, che diventa quasi impossibile da distinguere da una
	situazione problematica reale, ma non causata da attacco.
\end{itemize}
Molti sistemi utilizzano quindi dei filtri anti-spam, allo scopo di evitare
richieste da potenziali attaccanti.

Descriviamo ora varie tipologie di attacchi di tipo Denial of Service, che 
sfruttano la fase di handshake del protocollo TCP:
\begin{itemize}
\item\textbf{Denial of Service tramite Syn-Flooding} Tramite questa tecnica, si mira 
	ad inviare alla vittima un pacchetto IP modificato, in modo che sia
	cambiato l'indirizzo dell'attaccante/mittente con quello di un altro host
	casuale nella rete (che può quindi essere anche non esistente), 
	tramite una tecnica di \textit{spoofing}; quest'ultimo  non risponderà allo 
	\texttt{SYN-ACK} della vittima che avrà ricevuto un \texttt{SYN}, con la quale 
	non aveva iniziato  ad instaurare nessuna comunicazione. In questo modo,
	finché non scadrà il nostro timeout, la vittima terrà occupata una
	sua risorsa, occupando quindi la memoria del suo sistema. 
	
	Essendo quest'attacco di tipo flooding, l'attaccante potrebbe inviare
	sempre nuove richieste di tipo \texttt{SYN}, effettuando spoofing nel pacchetto
	e modificando il mittente con differenti mittenti casuali. In questo modo
	si vogliono impedire anche eventuali collegamenti legittimi nei 
	confronti del sistema attaccato, rendendo indisponibile il servizio.
\item \textbf{Syn-flooding reflector}.  Con questa tecnica invece effettuiamo l'invio
	del pacchetto, sul quale è stato effettuato spoofing e modificato il
	campo mittente con quella vittima, ad un host casuale intermedio che,
	ricevendo il \texttt{SYN}, invierà lo \texttt{SYN-ACK} alla vittima. In questo
	caso verranno replicati i riceventi, che sono ignari del vero attacco
	in corso dal vero mittente che però non conoscono, i quali invieranno 
	tutti la lororisposta al sistema attaccato. 
	
	In questo caso non verranno quindi consumate risorse del
	sistema attaccato, ma bensì verrà occupato tutto il canale di comunicazione
	verso questo: anche in questo modo si mina la fruibilità del servizio,
	in quanto si blocca l'accesso del canale di comunicazione.
\item\textbf{Bot-net (DDos)}. Tramite questa tecnica si ha un computer ``padrone'', che ha 
	la possibilità di inviare dei comandi ad una rete costituita da computers detti
	\textit{zombie}(s), in modo da orchestrare l'attacco verso una stessa vittima:
	questo è possibile perché gli zombie(s) che costituiscono la bot-net,
	sono sempre in ascolto del ``padrone''.  Tali zombie si ottengono
	tramite l'installazione inconscia di software, che opera all'interno di
	computer casalinghi senza dare segni evidenti agli utenti dei sistemi
	infettati.
\end{itemize}


In pratica, è quasi impossibile essere in grado di prevenire o/e difendersi
da un attacco di tipo DoS: non esiste quindi un unico tipo di difesa possibile
in merito, anche perché le tecnologie di attacco sono in continua evoluzione,
parallelamente alle tecniche di difesa. All'interno dei routers possono comunque
essere implementate delle tecniche di difesa contro DoS, anche se raramente 
sono divulgate le caratteristiche di previsione degli attacchi, poiché in questo
campo la security by secrecy è l'unica arma possibile per non divulgare il modo
con il quale viene sventato l'attacco, e per non minare il sistema che viene
commercializzato.
