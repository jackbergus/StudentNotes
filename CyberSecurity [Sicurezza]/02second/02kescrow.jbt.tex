\section{Key Escrow and Secret Sharing}

Con il termine \textbf{escrow} si vuole intender un concetto che va oltre al 
semplice deposito di un bene (in questo caso la chiave): con questo termine 
si intende infatti la consegna di questo ad un terzo, dal quale possiamo 
riottenerlo in base ad una condizione che si è verificata. Vedremo ora quali 
problemi ponga la custodia di tali ``segreti'' (le chiavi), in base all'affidamento
ad una o più persone. 


Supponiamo inizialmente che la chiave o una password per accedere ad alcune
risorse condivise sia conosciuta da una 
sola persona; questi possono essere la chiave privata di crittografia 
asimmetrica, o la chiave di crittografia simmetrica, o dei dati presenti
all'interno di un calcolatore che, non verranno mai comunicati, ma che 
non vogliamo che siano accedibili dall'esterno. Vogliamo inoltre che l'utente possa accedere
in un modo trasparente ai dati, senza conoscere necessariamente tale segreto.


Se il segreto rimane custodito unicamente da quella persona, ci si pone il problema di come 
accedere alle informazioni in caso di sua assenza, 
se per una serie di motivi queste non sono conseguentemente più disponibili. 
Vorremo inseguito che anche altri individui autorizzati possano accedere a quei dati,
anche in mancanza del personale irreperibile. Introdurremo quindi questo problema per gradi.


\subsection{Security officer}
Una possibile soluzione al problema si potrebbe avere con un \textbf{secutiry 
officer}, che custodisce per l'azienda il segreto che è stato affidato
ad un altro utente, in modo tale che si possano sempre recuperare i dati anche 
in mancanza di quel preciso utente. Questo segreto potrà essere inserito 
all'interno di un qualsiasi supporto, di modo che il security officier non conosca tale segreto
se non in quella forma: ciò potrebbe essere utile per effettuare 
controlli sul sistema, in modo da controllare le operazioni svolte dal sistema
quando l'utente privilegiato è assente, dando a quest'ultimo 
eventualmente un avviso, qualora il controllore agisca con il suo stesso segreto
quando invece è in servizio. 


In questo modo, tuttavia, si riavrebbe da capo lo stesso problema: la mancanza
di questo security officier potrebbe causare a sua volta la rovina dell'intero
sistema: si potrebbe quindi dividere la nostra chiave in $n$ parti ed affidarla
a rispettivi security officiers, criptando quei pezzi con le loro chiavi pubbliche.
In questo modo vogliamo far diminuire esponenzialmente la probabilità che,
alla mancanza di un agente, il segreto non sia più custodibile. 


Possiamo inoltre ben vedere che non potremo dividere esattamente in $k$ parti 
la chiave, poiché in questo modo sarebbe possibile ricostruire con poco 
sforzo la parte rimanente: se $k=2$, saremmo a conoscenza di 
metà del segreto, facendo così scendere la possibilità di riconoscimento della
chiave da una sua parte da $p^k$ a $p^{\frac{1}{2}k}$ (con questo appunto
intendiamo una qualsiasi suddivisione della risorsa in senso effettivo, ovvero
un suo partizionamento).


Passiamo ora ad analizzare un caso immediato di ``suddivisione'', con $k=2$:
in quanto sappiamo che $x \oplus x \oplus y = x\oplus y \oplus x = y$, allora
generato un valore random \texttt R della stessa lunghezza del segreto \texttt S,
possiamo consegnare ad un agente \texttt R ed all'altro \texttt R$\oplus$\texttt S. In questo
modo, solamente con la compresenza dei due, si potrà ricostruire il segreto iniziale.
In questo modo inoltre non si rileva più necessario mantenere il segreto iniziale
\texttt S. Si può dimostrare inoltre che questo metodo è di fatti \textbf{confidenziale},
in quanto nessun agente trae alcun vantaggio dalla conoscenza della loro parte
di chiave; sebbene questo sia evidente per il detentore di \texttt R, questo risulta
meno ovvio per \texttt R$\oplus$\texttt S, in quanto ad ogni modo si intravede un legame
con \texttt S ma, essendo ottenuto \texttt R con una stringa casuale, ci si ricondurrebbe
alla ricostruzione della stringa casuale \texttt R, esattamente come nel cifrario
perfetto \textit{one-time-pad}. Possiamo generalizzare questo comportamento 
tra $n$ agenti assegnando ad $n-1$ agenti i valori random $\mathtt R_1\dots \mathtt R_{n-1}$,
ed assegnando all'ultimo il segreto $\mathtt R_1\oplus\dots \mathtt R_{n-1}\oplus \mathtt S$.
Tuttavia in questo modo non garantiamo la disponibilità, in quanto dobbiamo 
ora garantire che \textbf{tutti} i detentori del segreto siano sempre presenti,
peggiorando quindi la probabilità che, alla mancanza di uno, il segreto risulti
irrecuperabile. 


Segue necessariamente che, per garantire anche la \textbf{disponibilità}, dobbiamo
indebolire lo schema inizilae, in modo che per conoscere il segreto non siano
necessari tutti gli $n$ agenti, ma che siano presenti in un numero $t<n$
utilizzando uno schema a soglia (\textit{theresold scheme}): in questo modo si impone
che siano necessari solamente $t$ utenti per poter accedere al segreto.


Ciò è risolubile non generando gli $\mathtt R_i$ in modo casuale, ma dei polinomi
casuali di grado $t-1$:
\[g(x)=(\sum_{i=1}^{t-1} a_ix^i+\mathtt S)\text{ mod  } p\]
dove $p$ è un numero \textit{primo} più grande dei valori costanti scelti arbitrariamente
$a_i$, ovvero maggiore del $\max \Set{a_i}_{i< t}$; in seguito ad
ogni agente verrà associato un segreto:
\[\forall i\leq n. \mathtt S_i = g(i)\]
in questo modo garantiamo che la curva $g(i)$ sia ricostruibile conoscendo 
solamente $t$ valori che stanno su di essa.
