\chapter{Gestione delle chiavi}
\section{Gestione delle chiavi pubbliche - Certificati}
In quanto non dobbiamo trattare la gestione delle chiavi private nella crittografia
asimmetrica, in quanto queste rimarranno sempre, dovremmo porre l'attenzione
sulla divulgazione delle chiavi pubbliche. 

Se un utente \texttt A effettua un annuncio pubblico della sua chiave pubblica (es.)
per poter raggiungere anche un possibile interlocutore \texttt B, questa
forma di comunicazione non richiede la presenza di intermediari, ma si ha il
problema di un possibile \textit{man-in-the-middle attack}: se lo comunichiamo
direttamente all'interno di un canale di comunicazione, un ascoltatore malevolo \texttt X
potrebbe cambiare tale chiave con la propria di modo che, quando \texttt B vuole
aver garanzia di comunicare con \texttt A, utilizzerà la chiave pubblica che 
crederà di \texttt B ma che è in realtà di \texttt X, di modo che \texttt X riceverà la comunicazione
da \texttt B che crederà di parlare con \texttt A. Questo potrebbe anche provocare il
seguente disagio: entrambi i lati della comunicazione potrebbero credere di 
comunicare con il diretto interessato, senza aver modo di controllare che,
effettivamente, la comunicazione è mediata da \texttt X. Si può quindi vedere che 
ciò è stato possibile perché \textbf{la comunicazione non garantisce la 
confidenzialità}.


Una soluzione possibile è quella di ritornare ad un ente centralizzato e 
fidato, che è responsabile della pubblicazione delle chiavi: tuttavia anche in
questo contesto abbiamo che dei possibili ``elenchi pubblici'' possano essere
falsificati, violandone quindi le informazioni. Inoltre, dobbiamo 
assicurarci che anche la registrazione delle informazioni debba avvenire in un
modo sicuro, altrimenti l'informazione potrebbe essere (es.) compromessa persino
qualora \texttt A chieda all'ente di aggiornare l'informazione; inoltre la stessa
directory dove possono esssere presenti queste chiavi può essere al contempo non
sicura, compromettendo tutti i dati contenuti al suo interno. 

Possiamo quindi concludere che una soluzione praticabile è quella dei \textbf{certificati},
all'interno dei quali vengono trasportate le chiavi pubbliche: queste sono
quindi sono rese autentiche dalla \textbf{firma} dell'autorità garante:
\begin{itemize}
\item In questo modo si verifica l'eliminazione dell'attacco \textit{man-in-the-middle},
	in quanto in questo modo non è possibile modificare la chiave pubblica
	nel certificato senza alterarlo
\item Si garantisce l'attualità delle chiavi, mitigando il problema del furto della
	chiave privata
\end{itemize}
\bigskip

Dobbiamo ora generare un oggetto che possa prevenire l'attacco \textit{man-in-the-middle},
ovvero dobbiamo fare in modo, durante lo scambio delle chiavi, che un terzo 
possa spacciarsi per un altro utente del sistema, cambiando semplicemente la 
chiave pubblica: come abbiamo già visto, possiamo avere questa garanzia tramite un
\textbf{certificato} rilasciato (es.) da un'autorità garante (che applica la sua 
firma sul documento) verso la quale riponiamo
fiducia: questo deve quindi incaricarsi di garantire che l'oggetto venga
consegnato direttamente nelle mani di chi lo richiede, e deve essere in grado
di accertare l'identità dell'individuo, che deve essere presente (assieme alla
sua chiave pubblica) all'interno del certificato. 


Per fare un esempio pratico, all'interno di un ambito ristretto, si può ritenere
la mail del richiedente una garanzia sufficiente per ottenere un certificato,
in quanto si ha la garanzia che solamente quel possibile destinatario sia in 
grado di poter leggere la mail (garanzia d'identità \textit{out of band}). Tuttavia
ciò non si rivela più sufficiente qualora siano richiesti ulteriori dati sul 
soggetto.
\medskip

Definiamo inoltre \textbf{ PKI} (\textit{Public Key Infrastructure}) l'insieme dei servizi
logicamente indipendenti fornibili da una rete di:
\begin{itemize}
\item \textbf{Registration Authority}: effettua l'ottenimento di una richiesta di un
	certificato da parte di un utente.
\item \textbf{Certification Authority}: effettua il controllo sull'identità 
	del richiedente, ed ha quindi il compito di rilasciare i certificati; 
	si può realizzare tramite procedure automatizzate
	e non richiedendo il contatto diretto con l'interessato.
\end{itemize}
In genere le \textit{PKI} contengono un insieme di servizi e di protocolli per
fornire (tramite il processo di certificazione), immagazinare, \textbf{validare} 
(ovvero, controllare che il certificato sia da considerarsi tutt'ora valido)
e revocare certificati. 
I \textit{certificate servers}, che contengono al loro interno database di chiavi
accessibili dalla rete, permettono ai loro utenti di effettuare una richiesta
di inserimento di un nuovo certificato personale (processo di \textbf{certificazione}), 
o di ottenere quello di qualcun altro. 

\subsection {X.509}
Questo certificato si basa sul concetto di ``trust'', che è presente anche 
nell'ambito umano: chi accetta come valido un certificato come un passaporto,
implicitamente ripone fiducia anche nell'ente, detto \textit{Certification Authority} 
che lo ha rilasciato. 

La \textit{X.509} è un ente di standardizzazione che utilizza un certificato con particolari
campi:
\begin{itemize}
\diam \texttt{Subject}: \textit{Distringuished Name} (Common Name, Organization or Company, 
	Oraganizational Unit, City/Locality, State/Province, Country (ISO code)),
	\textit{Public Key}
\diam \texttt{Issuer} (colui che ha rilasciato il certificato): \textit{Distinguished 
	name}, \textit{Signature} 
\diam \texttt{Validity}: \textit{Not Before Date}, \textit{Not After Date}
\diam \texttt{Administrative Info}: \textit{Version}, \textit{Serial Number}
\diam \texttt{Extended Info}:  \dots
\end{itemize}
Come possiamo vedere dal campo \texttt{Subject}, non è sufficiente avere il solo
\textit{Common Name}, in quanto in questo modo non si potrebbero gestire
i casi di omonimia. Inoltre, aggiungendo le ulteriori caratteristiche di cui
sotto, uno stesso soggetto può avere diversi certificati, in base al contesto
nel quale li deve utilizzare.

Dobbiamo inoltre notare che, all'interno di uno stesso
\textit{PKI}, possono essere presenti anche più \textit{Certification Authority} (CA)
decentralizzati, creando in questo modo una gerarchia diffusa di fiducia all'interno
del sistema. Inoltre, uno stesso CA può essere delegato da un PKI per certificare
altri CA: in questo modo si ha a che fare con una gerarchia di CA, fino ad
arrivare agli utenti, che costituiscono le foglie del sistema. Si può inoltre 
sottolineare come, uno stesso ente, possa rilasciare 
anche più tipologie di certificati, ad esempio, ad uno stesso utente.
\medskip

Supponiamo ora di creare un certificato nella forma:
\begin{center}
<Distinguished Name U, Public Key U, Signature CA X>
\end{center}
Per convincermi che questo certificato è valido, devo essere certo che anche la
firma di X sia valida: dovrò quindi ottenere un ulteriore certificato per X 
firmato da un altro CA, strutturato in modo analogo a sopra ma con la firma di
un altro CA; questa catena di certificati prima o poi dovrà essere interrotta
con un'autocertificato di un'entità \textit{Root Authority}, che ovvero pone la sua stessa
firma sul suo Distinguished name e la sua chiave pubblica.

In genere questi certificati sono cablati all'interno di applicativi, di modo
che possiamo conoscere a priori chi ritenere una Root Authority affidabile; in questo 
modo, è sufficiente che arrivi un certificato di una CA rilasciato da un Root
noto, per ottenere garanzia sulla chiave pubblica in esso contenuta.
\medskip

In seguito, dobbiamo considerare i casi con i quali possiamo ottenere la validazione
o la revoca di un certificato:
\begin{description}
\item [Validazione] tale operazione è necessaria in quanto l'informazione di un 
	certificato può cambiare col tempo, e quindi abbiamo l'esigenza di 
	controllare se tali certificati sono da considerarsi o meno \textit{aggiornati}
	ed \textit{autentici}. Si può effettuare questa operazione \textbf{offline}, ma a questo
	punto si può effettuare il controllo unicamente andando ad analizzare 
	il certificato, e controllando il suo periodo di validità; l'operazione
	è inoltre fattibile anche \textbf{online}, ed in questo modo si può sempre 
	chiedere ad ogni istante al CA che l'ha rilasciato se il certificato 
	corrente è valido (uno strumento simile si potrebbe avere nel controllo
	offline tramite una lista dei certificati non più validi, anche se questo
	elenco non può essere tenuto in costante aggiornamento).
\item [Revoca] La revoca del certificato può avvenire qualora si reputi che 
	le informazioni in esso contenute, non sono più valide. Questo può avvenire
	(\emph{i}) perché la firma del CA, per un motivo grave, non è più valida,
	(\emph{ii}) perché l'utente ha cambiato la sua chiave segreta - e
	contemporaneamente quella pubblica, (\emph{iii}) le informazioni riguardanti
	l'utente cambiano. La revoca del certificato in modalità \textbf{offline}
	diventa critica, mentre il problema risulta triviale in modalità
	\textbf{online}. 
\end{description}

\subsection{PGP (Pretty Good Privacy)}
Il sistema PGP introduce un nuovo sistema per l'attribuzione dell'affidabilità
in un modello non gerarchico: in quello gerarchico invece, abbiamo che una root authority
può assumere questo privilegio unicamente tramite una conoscenza \textit{out of band},
la quale implica una conoscenza di come questo abbia operato nel passato.

Questo sistema invece, da la possibilità ad ognuno di autocerficarsi, garantendo
inoltre relazioni di fiducia tra utenti, di modo da creare un grafo di relazioni
di affidabilità, che può cambiare contestualmente al cambiamento delle relazioni
tra utenti. Tale pacchetto software (disponibile anche freeware), consente l'uso
di:
\begin{itemize}
\diam Generazione, e gestione delle chiavi
\diam Cifratura e decifratura, firma e verifica di documenti digitali
\diam $\Rightarrow$ gestione di una VPN sicura.
\diam Creazione di Self-Decrypting Arhchives
\diam Cancellazione permanente di file, directory e spazio sul disco.
\end{itemize}

Il sistema di mantenimento delle chiavi, prevede di conservare per noi criptate
secondo una \textit{passphrase} ben conosciuta, le chiavi pubbliche e private di
un utente che, al contrario delle password numeriche, possono essere ricordate
facilmente dall'utente, ma sono ottenibili difficilmente tramite un attacco di
forza bruta. In questo modo, l'utente può anche generare ma non vedere mai le
sue chiavi pubbliche e private, ma è necessario che non si dimentichi tale
\textit{passphrase}.
\medskip

Possiamo ora vedere quali azioni dovrà compiere il mittente per inviare un 
documento: una volta compresso il documento, egli genererà una chiave di 
sessione con la quale cripterà il compresso, codificandolo assieme a detta chiave 
con quella pubblica del destinatario. Quest'ultimo potrà aprire il suo documento
tramite la sua chiave privata, che gli permetterà di accedere alla chiave di 
sessione per poi poter decifrare il messaggio a lui solo destinato.

Possiamo notare che, tramite la compressione, abbiamo un rafforzamento del 
protocollo poiché, in questo modo, si può evitare l'attacco statistico del
plaintext: comprimendo il testo apparirà di natura casuale, in quanto 
non si mantiene l'ordinamento del documento, poiché le lettere verranno 
mappate in un modo differente. Altro vantaggio che si ottiene tramite la 
compressione è la ridotta dimensione del messaggio da inviare a destinazione.

All'interno dei cifrari a chiavi pubbliche, in genere queste chiavi vengono sempre
comunicati tramite certificati, che sono strutturati in un modo differente rispetto
a \textit{X.509}:
\begin{itemize}
\diam Chiave pubblica di U e caratteristiche della chiave (lunghezza, algoritmo,
	data di creazione e sua durata)
\diam Informazioni sull'identità di U
\diam Auto-firma di U
\diam Indicazione dell'algoritmo simmetrico di codifica
\end{itemize}
Inoltre il sistema PGP è in grado di gestire (e quindi riconoscere) il sistema
X.509, anche se non è supportata l'interoperabilità dei due sistemi, nel senso
del riconoscimento nei certificati PGP quelli X.509.
Mettiamo ora in luce le differenze tra questi due sistemi di Key Management:
\begin{itemize}
\item In PGP non esiste il concetto di ``Registration Authority'', come invece
	avviene per l'altro standard;
\item In PGP il certificato nasce  autofirmato, mentre in X.509 viene firmato
	dalla suddetta ``Registration Authority'';
\item In PGP possono essere ammesse identità multiple per uno stesso utente, 
	mentre X.509 deve essere in grado di identificare singolarmente ciascuna
	di quelle entità per evitare l'ambiguità
\item Si ripone fiducia nei confronti di un certificato non solo dal numero di
	firme in esse presenti, ma anche da quante firme sono presenti nei
	certificati dei firmatari.
\end{itemize}
Per conesntire le identità multiple, all'interno di uno stesso certificato
possono essere presenti coppie multiple chiavi-identità, ciascuna delle volte
firmate un numero non precisato di volte e ciascuna con ruoli differenti, e
utilizzi in differenti contesti.


Il certificato quindi nasce già da subito come autocertificato, e a questo 
punto ogni utente diventa in un certo modo come ``autority'' di sè stesso,
senza essersi rivolto ad alcuna autorità; per testare l'autorevolezza del certificato,
verrano esposte delle altre firme all'interno dello stesso di altri utenti, 
alle quali si può assegnare i seguenti gradi di attendibilità:
(\textit i) \textbf{complete} trust, (\textit{ii}) \textbf{marginal} trust e (\textit{iii})
\textbf{untrusted}. Inoltre, all'interno di questo contesto, dobbiamo essere in 
grado di dividere la \textbf{fiducia} che riponiamo su di un utente, dalla 
\textbf{validità} del certificato che esso utilizza: potrei riporre fiducia in un
certificato perché esso mi viene presentato da quella persona in persona,
attestando quindi la validità del certificato, e sostituendo in questo modo
la certificazione di un singolo utente. Possiamo quindi distinguere tre modelli 
differenti di affidabilità:
\begin{description}
\item[Direct Trust] ottengo la fiducia solamente delle persone con le quali
	interagisco direttamente
\item[Hiearchical Trust] la fiducia viene riposta nei confronti di un ente
	emittente in modo assoluto, tramite una conoscenza out-of-band della
	sua affidabilità
\item[Web of Trust] Si viene a creare un grafo arbitrario di affidabilità.
	Non esistendo una CA, tutti potranno essere CA in quanto tutti hanno la
	possibilità di firmare i certificati presenti. Inoltre, il primo 
	utente che mette una firma in un certificato viene definito come
	\textbf{introducer}, ed ognuno può considerare più o meno affidabile quel
	certificato, in base alla firme delle entità presenti all'interno di
	quel certificato. La struttura di una ``rete'' è anche dovuta all'assenza
	di autorità centrali di certificazione. Tuttavia le relazioni del grafo
	nè sono da considerarsi simmetriche, nè sono da considerarsi transitive,
	in quanto non sempre la fiducia, anche nella vita reale, è una relazione
	di quel tipo, e contemporaneamente quei valori settati all'interno del 
	certificato per ogni firma, possono cambiare con il tempo.
\end{description}

Abbiamo inoltre che, se una persona riceve un'affidabilità del tipo \textit{Completely
trust}, essa diventerà come una ``CA'', mentre una chiave è considerata come
valida se almeno un utente l'ha indicato come \textit{Completely trusted}, o più di
due firme \textit{Marginally trusted.}

Possiamo fornire alcuni dati:
\textit{
\begin{itemize}
\item Le chiavi segrete da 80b sono equivalenti a quelle pubbliche da 1024b;
	le chiavi segrete da 1028b invece sono equivalenti a quelle pubbliche a
	3000b
\item In quanto le chiavi segrete a 56b oggi non sono sicure, e quelle a 128b 
	magari non lo saranno nel futuro coi computer quantistici, quelle a 256b
	sono ritenute sufficientemente sicure anche in un futuro lontato
\item Gli algoritmi a chiave privata utilizzata sono CAST, Triple-DES, IDEA ed AES;
	quelli a chiave pubblica sono DSS ed RSA, mentre come algoritmo di HASH si
	usa SHA1. 
\end{itemize}
}

