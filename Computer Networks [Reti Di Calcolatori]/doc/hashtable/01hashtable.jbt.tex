\chapter{Libhashtable}\label{cha:lht}
\begin{itemize}
\diam \textbf{File di inclusione}: \texttt{proxy/libhashtable.h}, \texttt{hashtable/libhashtable.h} 
\end{itemize}
Abbiamo deciso di implementare separatamente la hashtable rispetto al proxy, in
quanto costituisce un nucelo di codice a sè stante, indipendente dalle funzioni
del proxy in senso stretto: unica funzione che è strettamente legata al proxy è
\texttt{parseHead\_time} in \texttt{hashtable/hres.c}, in quanto è necessaria per 
valutare se il file presente in cache sia o meno scaduto.

Come già discusso, lo scopo di utilizzare una hash table è quello di effettuare 
un locking il più selettivo possibile e indipendente dall'architettura: di fatti,
la systemcall \texttt{flock} non è implementato in Cygwin, in quanto non appartenente
allo standard POSIX\footnote{\texttt{http://cygwin.com/ml/cygwin/1999-04/msg00026.html}},
mentre nei sistemi Windows è utilizzata \texttt{LockFileEx}\footnote{
\texttt{http://goo.gl/Yu031}}.

Tramite una tabella di hashing, possiamo conseguentemente effettuare il loking
in \textit{n} parti della cache, senza effettuare il lock sull'interno filesystem.
Si possono ridurre le collisioni aumentando la dimensione della hash, ma di 
massima si ridurrebbero solamente le performance del proxy, in quanto si 
verificherebbero maggiorni collisioni all'interno della tabella di hashing.

Tale hashtable, è implementata come un array di \textit{n} elementi, dove ciascuna
posizione è attribuibile ad un preciso valore di hashing: per quella stessa
posizione, sono inoltre indicati se sono presenti elementi o meno; se non è
presente alcun elemento, allora non verrà eseguita nessuna syscall, e verrà
ritornato un valore negativo. Se invece abbiamo che è presente un elemento al
suo interno, allora si deve controllare se questo esiste realmente nella data
posizione: questo è necessario in modo da verificare se tra quegli elementi è
presente anche la risorsa considerata. Tutte queste operazioni sono effettuate
all'interno di \texttt{hres.c}.

Riassumiamo ora brevemente il contenuto dei files rimanenti:
\begin{description}
\item[filesystem.c]: contiene interazioni base con il filesystem, non sempre
	legate alla hashtable, ma sempre legate al caching: queste funzioni
	o sono adoperate direttamente a livello di proxy (come \texttt{recursiveDelete}
	che effettua lo svuotamento della cache all'apertura del proxy), o sono
	adoperate all'interno della libreria
\item[fsys.c]: effettua operazioni basi sui files, tra le quali l'allocazione
	di un'area di memoria atta alla memorizzazione del contenuto del file.
\item[hash.c]: contiene le funzioni che permettono di operare sulla struttura
	dati di hashing in quanto tale, indipendentemente dal fatto che tale
	struttura dati sia utilizzata allo scopo di effettuare il locking sui files.
\item[lett\_scritt.c]: sono presenti le funzioni per effettuare il locking.
\end{description}
