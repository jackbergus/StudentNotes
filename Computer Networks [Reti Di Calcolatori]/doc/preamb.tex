%% preambolo standard dei miei documenti %%
\documentclass[a4paper,11pt,italian]{book}

\usepackage{tikz,xifthen}
\usepackage{pgf}
\usepackage{makeidx}
\usepackage{amsfonts}
\usepackage{amsmath,amsthm}
\usepackage{mathpazo}
\usepackage{booktabs}
\usepackage{tabularx}
\usepackage{rotating}
\usepackage{graphicx}
\usepackage[italian]{varioref}
\usepackage[tight,italian]{minitoc}
\usepackage{braket}
\usepackage[T1]{fontenc}
\usepackage[utf8]{inputenc}
\usepackage[a4paper]{geometry}
\geometry{verbose,tmargin=2cm,bmargin=3cm,lmargin=2cm,rmargin=2cm,headheight=2cm,headsep=1cm,footskip=1cm}
\usepackage{amsthm}
\usepackage{babel}
\PassOptionsToPackage{normalem}{ulem}
\usepackage{ulem}



\usepackage{listings} %­Per inserire codice
%\usepackage[usenames]{color} %­Per permettere la colorazione dei caratteri

\lstset{ %
  language=SQL,                % the language of the code
  showspaces=false,               % show spaces adding particular underscores
  showstringspaces=false,         % underline spaces within strings
  showtabs=false,                 % show tabs within strings adding particular underscores
  captionpos=b,                   % sets the caption-position to bottom
  breaklines=true,                % sets automatic line breaking
  breakatwhitespace=false,        % sets if automatic breaks should only happen at whitespace
  title=\lstname,                 % show the filename of files included with \lstinputlisting;
                                  % also try caption instead of title
}


\newcommand{\diam}{\item[$\diamond$]} % punto elenco diamante
\theoremstyle{plain}\newtheorem{teorema}{Teorema}[chapter]
\theoremstyle{plain}\newtheorem{corollario}{Corollario}[chapter]
\theoremstyle{plain}\newtheorem{define}{Definizione}
\makeatother


\usepackage{subfig}
\setcounter{secnumdepth}{9}
\setcounter{tocdepth}{9}
\usetikzlibrary{er}
\usepackage[unicode=true, pdfusetitle,
 bookmarks=true,bookmarksnumbered=true,bookmarksopen=true,bookmarksopenlevel=1,
 pdfborder={0 0 1},backref=false,colorlinks=false]
 {hyperref}

% \input{beghe} % contiene le mie funzioni di facilitazione
\newtheorem{theorem}{Teorema}
